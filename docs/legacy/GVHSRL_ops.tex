\documentclass[a4paper,12pt]{report}
\usepackage{cite}
\renewcommand{\bibname}{references}
\renewcommand{\abstractname}{Introduction}
\usepackage{hyperref}

\begin{document}

\title{HSRL startup procedure}
\maketitle

\newpage

\section{HSRL associated computers}

Access computers in Lidar Lab in FL1: 

HSRL control 

  ssh -a -XY -C hsrl@hsrl-router.eol.ucar.edu

Archiver 

  ssh -p 23 -a -XY -C hsrl@hsrl-router.eol.ucar.edu

\section{Startup Procedure}

On the HSRL electronics rack

\begin{itemize}

\item Turn on the thermal control power switch. 

\item Turn on Computer power switch (this is usually already on if HSRL is in FL lidar lab). 

\item Turn on Laser power switch. 

\end{itemize}

\noindent On the Photonics Ind Panel

\begin{itemize}

\item Power on switch. 

\item laser enable. 

\item shutter control on. 

\item LDD (laser diod driver) button 

\item MENU button then enter button twice to allow you to change the laser diode current 

\item Turn knob to set IS to 26.5 Amps. (laser currently needs replacing, current value to be updated.) 

\end{itemize}

\noindent On the HSRL control computer

\begin{itemize}

\item start\_all

\item make sure main shutter is open on webpage: 

\end{itemize}

\noindent On the HSRL archiver

\begin{itemize}

\item start\_all

\item make sure images are reaching the field catalog: http://catalog.eol.ucar.edu/operations/lidar

\end{itemize}



\noindent To check that everything is running properly you can run hsrl\_status. 

\noindent To turn off the HSRL reverse these steps using stop\_all instead of start\_all. 

\end{document}
